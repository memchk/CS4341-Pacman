\documentclass[12pt]{article}
\usepackage[utf8]{inputenc}
\usepackage[letterpaper, margin=1in]{geometry}
\usepackage{tabularx}
\usepackage{caption}

\setlength{\parindent}{0em}
\renewcommand{\arraystretch}{1.25}
% Title Block
\title{
    CS4341: Bec-Man System Design \\
    \large The MetaStable Flip-Flops}
\author{Carson Page \and Husna Chaudhary}
\date{Fall 2019}
\begin{document}
\begin{titlepage}
    \maketitle
\end{titlepage}
\tableofcontents
\newpage

\setlength{\parskip}{1em}

\section{System Listing}
\subsection{Defined Types}

We defined multiple new types to assist in development. These are replicated
commonly in the next sections.

\begin{tabularx}{\linewidth}{ ||c|c|X|| }
    \hline
    Type Name   & Width     & Description \\
    \hline
    s\_width\_t & clogb2(SCREEN\_WIDTH) & Generated type that fits the visible
    screen width. \\
    s\_height\_t & clogb2(SCREEN\_HEIGHT)  & Generated type that fits the visible
    screen height. \\
    coord\_t     &  s\_width\_t + s\_height\_t & XY Coordinate Pair. \\
    rgb\_t      & [23:0]    & Packed struct representing 8:8:8 RGB color. \\
    \hline
\end{tabularx}

\subsection{Module Descriptions}
Each used module is documented here with a high-level description and its
function. Detailed port listings follow in the next section.

\subsection{Input List}
Each module has its input(s) name, type, and description listed in a table
below. Inputs that are prevelent and share the same semantic meaning are in Table
\ref{tab:input_common}.

\begingroup
\captionof{table}{\textbf{Common Inputs}}
\label{tab:input_common}
\begin{tabularx}{\linewidth}{ ||c|c|X|| }
    \hline
    Name & Type & Description \\
    \hline
    i\_clk & logic & Global system clock tree. All logic is driven off of this
    or a derived clock from a PLL / MMCM module. \\
    i\_rst & logic & Active high reset. Module must reset registers and/or
    outputs to known base state on activation. \\
    i\_en   & logic & Active high clock enable. Registers part of the data path
    should only update when activated. Control registers or edge detect
    registers \emph{may} not be disabled on a low if necessary for proper function. \\
    \hline
\end{tabularx}
\endgroup

\newpage
\begingroup
\captionof{table}{\textbf{Video Timing Generator (vtg)}}
\label{tab:input_vtg}
\begin{tabularx}{\linewidth}{ ||c|c|X|| }
    \hline
    Name & Type & Description \\
    \hline
    ACTIVE\_WIDTH & parameter & The width of the visible area of the display. \\
    ACTIVE\_HEIGHT & parameter & The height of the visible area of the display.
    \\
    V\_FRONT\_PORCH & parameter & Vertical front porch of the timing spec in
    lines. \\
    V\_BACK\_PORCH & parameter & Vertical back porch of the timing spec in
    lines. \\
    V\_PULSE & parameter & Length of vertical sync pulse in lines. \\
    V\_POL & parameter & Defines if the module outputs a active high/low
    vertical sync pulse. \\
    H\_FRONT\_PORCH & parameter & Horizontal front porch of the timing spec in
    pixels. \\
    H\_BACK\_PORCH & parameter & Horizontal back porch of the timing spec in
    pixels. \\
    H\_PULSE & parameter & Length of horizontal sync pulse in pixels. \\
    H\_POL & parameter & Defines if the module outputs a active high/low
    horizontal sync pulse. \\
    \hline
\end{tabularx}
\endgroup

\begingroup
\captionof{table}{\textbf{Game State Engine (game\_state)}}
\label{tab:input_game_state}
\begin{tabularx}{\linewidth}{ ||c|c|X|| }
    \hline
    Name & Type & Description \\
    \hline
    i\_joystick & logic [3:0] & Joystick Input, lowest bit represents the left
    direction. Successive bits represent the next cardinal direction in a
    counter-clockwise manner. The zero vector represents no input. \\
    \hline
\end{tabularx}
\endgroup

\subsection{Output List} 
\end{document}